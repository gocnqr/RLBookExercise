
\documentclass[10pt,letterpaper]{article}



% Some useful packages
% math
\usepackage{amsmath}
\usepackage{amsfonts}
\usepackage{algorithm}
\usepackage[noend]{algpseudocode}
\usepackage{amssymb}
% pretty colors
\usepackage{color}
% nicer urls that break at the end of the page
\usepackage{url}
% every document needs images
\usepackage{graphicx}
\usepackage{setspace}
\usepackage{caption}
\usepackage{subcaption}


\newcommand{\e}{\mathbb E}
\newcommand{\R}{\mathbb{R}}
\newcommand\given[1][]{\:#1\vert\:}
\DeclareMathOperator*{\argmin}{arg\,min} % thin space, limits underneath in displays
\DeclareMathOperator*{\argmax}{arg\,max} % thin space, limits underneath in displays
\newcommand{\var}[1]{{\operatorname{\mathit{#1}}}}
\algdef{SE}[SUBALG]{Indent}{EndIndent}{}{\algorithmicend\ }%
\algtext*{Indent}
\algtext*{EndIndent}

\graphicspath{{./figures/}}   % where to look for images

%let's fiddle with the default margins to save some trees
%this makes the odd side margin go to the default of 1inch
\oddsidemargin 0.0in
%sets the textwidth to 6.5, which leaves 1 for the remaining right margin with 8 1/2X11inch paper
\textwidth 6.5in
% less white space, please!
\headheight 0.0in
% shift everything up
\topmargin -0.5in
\footskip .6in
% text should take up all but a 1'' margin
\textheight 9.0in

% Define some shortcuts for things I want to use.
% Use them like, for example:
% \begin{hypothesis}Lettuce causes brain damage.\end{hypothesis}
% Numbering & formatting will happen automatically.
\newtheorem{hypothesis}{Hypothesis}
\newtheorem{task}{Task}
\newtheorem{contribution}{Contribution}


% Shortcuts: allows you to use limited markup when editing/collaborating.
% \comment{This section needs to be rewritten.}
\def\ask#1{\textcolor{red}{\bf $\langle\langle$Question:\ #1$\rangle\rangle$}}
\def\comment#1{\textcolor{red}{\bf $\langle\langle$Comment:\ #1$\rangle\rangle$}}


% This imitates the Wikipedia ``Citation Needed'' text; use it as a temporary
% marker for things you need to cite.
\def\citationneeded{$^{\textcolor{blue}{\text{[citation needed]}}}$ }

% format et al.
\def\etal{\textit{et al.}}
\def\ie{\textit{i.e.}}
\def\eg{\textit{e.g.}}

\title{Attempts to exercise in Reinforcement Learning book Chapter 7}
\author{Mengliao Wang}


\begin{document}

% Generate Title Page
\maketitle


% this dumps the abstract on a front page all by itself.

\section*{Exercise 7.1: }
\label{7.1}

First of all, according to definition of $G_t$ and $G_{t:t+n}$ we have the following equations:

\begin{align*}
G_t &= R_{t+1} + \gamma R_{t+2} + ... + \gamma^{T-t-1}R_T\\
G_{t+n} &= R_{t+n+1} + \gamma R_{t+n+2} + ... + \gamma^{T-t-n-1}R_T\\
G_{t:t+n} &= R_{t+1} + \gamma R_{t+2} + ... + \gamma^{n-1}R_{t+n} + \gamma^nV_{t+n-1}(S_{t+n})
\end{align*}

From these equation we can have 

\begin{equation}
G_t = G_{t:t+n} -\gamma^nV_{t+n-1}(S_{t+n}) + \gamma^nG_{t+n}
\end{equation}

Then by applying equation 1, the difference between $G_t$ and $V_{t+n-1}(S_t)$ can be written as:

\begin{align}
G_t - V_{t+n-1}(S_t) &= G_{t:t+n} -\gamma^nV_{t+n-1}(S_{t+n}) + \gamma^nG_{t+n} - V_{t+n-1}(S_t) \\
&= [G_{t:t+n} - V_{t+n-1}(S_t)] + \gamma^n[G_{t+n} - V_{t+n-1}(S_{t+n})]\\
&= \delta_t + \gamma^n[\delta_{t+n} + \gamma^n[G_{t+2n} - V_{t+2n-1}(S_{t+2n})]]\\
&= \sum_{k=0}^{t+kn<T}\gamma^{kn}\delta_{t+kn}
\end{align}


\section*{Exercise 7.2: }
\label{7.2}


\section*{Exercise 7.3: }
\label{7.3}

A larger random walk task will make the simulated sequence significantlly longer, which allows us to run TD approach on a bigger $n$. If we change the number of states to be smaller, it will be beneficial for smaller $n$ values, since less simulated sequences will have more steps than $n$. However, I do not think changing the left-side outcome from 0 to -1 would make a differece in the best value of $n$ here.

\clearpage

\end{document}
